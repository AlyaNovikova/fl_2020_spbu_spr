\documentclass[12pt]{article}

\usepackage{cmap}
\usepackage[T2A]{fontenc}
\usepackage[utf8]{inputenc}
\usepackage[russian]{babel}
\usepackage{graphicx}
\usepackage{amsthm,amsmath,amssymb}
\usepackage[russian,colorlinks=true,urlcolor=red,linkcolor=blue]{hyperref}
\usepackage{enumerate}
\usepackage{datetime}
\usepackage{fancyhdr}
\usepackage{lastpage}
\usepackage{color}
\usepackage{verbatim}
\usepackage{tikz}
\usepackage{epstopdf}
\usepackage{xifthen}

\def\NAME{}
\def\DATE{}
\def\CURNO{\NO\t{7}}
 
\parskip=0em
\parindent=0em

\sloppy
\voffset=-20mm
\textheight=235mm
\hoffset=-25mm
\textwidth=180mm
\headsep=12pt
\footskip=20pt

\setcounter{page}{0}
\pagestyle{empty}

\DeclareSymbolFont{extraup}{U}{zavm}{m}{n}
\DeclareMathSymbol{\heart}{\mathalpha}{extraup}{86}
\newcommand{\N}{\mathbb{N}}   \newcommand{\R}{\mathbb{R}}   \newcommand{\Z}{\mathbb{Z}}   \def\EPS{\varepsilon}         \def\SO{\Rightarrow}          \def\EQ{\Leftrightarrow}      \def\t{\texttt}               \def\c#1{{\rm\sc{#1}}}        \def\O{\mathcal{O}}           \def\NO{\t{\#}}               \renewcommand{\le}{\leqslant} \renewcommand{\ge}{\geqslant} \def\XOR{\text{ {\raisebox{-2pt}{\ensuremath{\Hat{}}}} }}
\newcommand{\q}[1]{\langle #1 \rangle}               \newcommand\URL[1]{{\footnotesize{\url{#1}}}}        \newcommand{\sfrac}[2]{{\scriptstyle\frac{#1}{#2}}}  \newcommand{\mfrac}[2]{{\textstyle\frac{#1}{#2}}}    \newcommand{\score}[1]{{\bf\color{red}{(#1)}}}

\def\makeparindent{\hspace*{\parindent}}
\def\up{\vspace*{-0.3em}}
\def\down{\vspace*{0.3em}}
\def\LINE{\vspace*{-1em}\noindent \underline{\hbox to 1\textwidth{{ } \hfil{ } \hfil{ } }}}


\lhead{Александра Новикова}
\chead{}
\rhead{\DATE}
\renewcommand{\headrulewidth}{0.4pt}

\lfoot{}
\cfoot{\thepage\t{/}\pageref*{LastPage}}
\rfoot{}
\renewcommand{\footrulewidth}{0.4pt}

\newenvironment{MyList}[1][4pt]{
  \begin{enumerate}[1.]
  \setlength{\parskip}{0pt}
  \setlength{\itemsep}{#1}
}{       
  \end{enumerate}
}
\newenvironment{InnerMyList}[1][0pt]{
  \vspace*{-0.5em}
  \begin{enumerate}[a)]
  \setlength{\parskip}{#1}
  \setlength{\itemsep}{0pt}
}{
  \end{enumerate}
}

\newcommand{\Section}[1]{
  \refstepcounter{section}
  \addcontentsline{toc}{section}{\arabic{section}. #1} 
{\LARGE \bf #1} 
  \vspace*{1em}
  \makeparindent\unskip
}
\newcommand{\Subsection}[1]{
  \refstepcounter{subsection}
  \addcontentsline{toc}{subsection}{\arabic{section}.\arabic{subsection}. #1} 
  {\Large \bf \arabic{section}.\arabic{subsection}. #1} 
  \vspace*{0.5em}
  \makeparindent\unskip
}

\newlength{\ShiftLength}\setlength{\ShiftLength}{1.8em}
\newcommand{\leftLabel}[2][]{\ifthenelse{\equal{#1}{}}{{\hspace*{-\ShiftLength}\makebox[0pt][r]{\color{black}{#2}}\hspace*{\ShiftLength}}}{{\hspace*{-\ShiftLength}\makebox[0pt][r]{\color{#1}{#2}}\hspace*{\ShiftLength}}}}

\newenvironment{code}{
  \VerbatimEnvironment

  \vspace*{-0.5em}
  \begin{minted}{c}}{
  \end{minted}
  \vspace*{-0.5em}

}

\newenvironment{smallformula}{
 
  \vspace*{-0.8em}
}{
  \vspace*{-1.2em}
  
}
\newenvironment{formula}{
 
  \vspace*{-0.4em}
}{
  \vspace*{-0.6em}
  
}

\definecolor{dkgreen}{rgb}{0,0.6,0}
\definecolor{brown}{rgb}{0.5,0.5,0}
\newcommand{\red}[1]{{\color{red}{#1}}}
\newcommand{\dkgreen}[1]{{\color{dkgreen}{#1}}}

\begin{document}

\pagestyle{fancy}
\begin{MyList}[8pt]
	
 \item[2)] 
	\begin{MyList}[8pt]
	 \item[0.0] Изначальная грамматика: \\
		\begin{itemize}
			\item $S \to R S \ | \ R$
			\item $R \to a S b \ | \ c R d \ | \ a b \ | \ c d \ | \ \epsilon$
		\end{itemize}
	 \item[0.1] Давайте вначале избавимся от смешанных и длинных терминальных правил : \\
	 	\begin{itemize}
	 		\item $S \to R S \ | \ R$
	 		\item $R \to A S B \ | \ C R D \ | \ A B \ | \ C D \ | \ \epsilon$
	 		\item $A \to a; \ B \to b; \ C \to c; \ D \to d$
	 	\end{itemize}
 	\item[1.] Избавляемся от длинных правил: \\
 		\begin{itemize}
 			\item $S \to R S \ | \ R$
 			\item $R \to A B_1 \ | \ C D_1 \ | \ A B \ | \ C D \ | \ \epsilon$
 			\item $A \to a; \ B \to b; \ C \to c; \ D \to d$
 			\item $B_1 \to S B; \ D_1 \to R D$
 		\end{itemize}
 	\item[2.] $\epsilon$-правила \\
 		Нетерминалы S и R являются $\epsilon$-порождающими, поэтому для них добавляем нужные правила: \\
		\begin{itemize}
			\item $S \to R S \ | \ R \ | \ \epsilon$
			\item $R \to A B_1 \ | \ C D_1 \ | \ A B \ | \ C D \ | \ \epsilon$
			\item $A \to a; \ B \to b; \ C \to c; \ D \to d$
			\item $B_1 \to S B \ | \ B; \ D_1 \to R D \ | \ D$
			
		\end{itemize}
		Убираем $\epsilon$-правила: \\
		\begin{itemize}
			\item $S \to R S \ | \ R \ | \ \epsilon$
			\item $R \to A B_1 \ | \ C D_1 \ | \ A B \ | \ C D$
			\item $A \to a; \ B \to b; \ C \to c; \ D \to d$
			\item $B_1 \to S B \ | \ B; \ D_1 \to R D \ | \ D$
		\end{itemize}
	\item [3.] Новое стартовое: \\
		\begin{itemize}
			\item $S_1 \to S \ | \ \epsilon$
			\item $S \to R S \ | \ R$
			\item $R \to A B_1 \ | \ C D_1 \ | \ A B \ | \ C D$
			\item $A \to a; \ B \to b; \ C \to c; \ D \to d$
			\item $B_1 \to S B \ | \ B; \ D_1 \to R D \ | \ D$
		\end{itemize}
	\item [4.] Убираем унарные правила, замыкая цепочки: \\
		\begin{itemize}
			\item $S_1 \to R S \ | \ A B_1 \ | \ C D_1 \ | \ A B \ | \ C D \ | \ \epsilon$
			\item $S \to R S \ | \ A B_1 \ | \ C D_1 \ | \ A B \ | \ C D$
			\item $R \to A B_1 \ | \ C D_1 \ | \ A B \ | \ C D$
			\item $A \to a; \ B \to b; \ C \to c; \ D \to d$
			\item $B_1 \to S B \ | \ b; \ D_1 \to R D \ | \ d$
		\end{itemize}
	\item [5.]
	Получилась нормальная форма Хомского!
 		
	\end{MyList} 
 
 
 \item[3)] Этот язык является контекстно-свободным, предъявим КС-грамматику: \\
 \begin{itemize}
 	\item $S \to aaZ \ | \ aZb \ | \ bbT$
 	\item $Z \to aaZ \ | \ aZb \ | \ bbT$
 	\item $T \to bbT$
 	\item $Z \to \epsilon$
 	\item $T \to \epsilon$
 \end{itemize}
Нетерминированные состояния - это $S, Z, T$, стартовое состояние - $S$ \\
\\
1. Докажем, что мы получаем все слова вида $a^n b^m$, где m + n > 0, (m + n) делится на 2 \\
 \begin{itemize}
	\item Если букв $a$ чётное количество, то так как сумма тоже чётная, то и букв $b$ чётное количество, значит можно просто сделать так: \\
	$S \to aaZ \to aaaaZ \to ... \to a^nZ \to a^nbbT \to a^nbbbbT \to ... \to a^nb^m$ \\
	($n$ и $m$ чётные, значит можно добавлять по две буквы) \\
	
	\item Если $n$ нечетное, то так как сумма чётная, то и $m$ нечетное \\
	Тогда можно сделать так: \\
	$S \to aaZ \to ... \to a^{n - 1}Z \to a^nZb \to a^nbbbT \to ... \to a^nb^m$ \\
	То есть мы $(n - 1) / 2$ раз применяем правило $Z \to aaZ$, один раз правило $Z \to aZb$ и правило $Z \to bbT$, и $(m - 1) / 2$ раз правило $T \to bbT$ \\

\end{itemize}
2. Докажем, что мы не получаем ничего лишнего \\
 \begin{itemize}
 	\item Пустое слово получить невозможно, так как из стартового символа $S$ все переходы сразу дают хотя бы 1 букву \\
 	\item Все слова имеют вид $a^nb^m$, так как во всех правилах буквы $a$ левее всего. И видно, что буква $b$ не может оказаться левее буквы $a$ \\
 	Формально, можно доказать по индукции по количеству сделанных переходов. \\
 	Утверждение: все буквы $a$ находятся левее всех$Z, T$ и $b$ \\
 	База понятна, переход: все наши правила не нарушают это утверждение. \\
 	\item Осталось доказать, что $m + n$ чётное \\
 	Но это просто инвариант. Изначально их чётное, после каждого правила их в сумме остаётся чётное число. Ч.Т.Д.
 	
 \end{itemize}
 

\end{MyList} 
\end{document}
