\documentclass[12pt]{article}
\usepackage[left=2cm,right=2cm,top=2cm,bottom=2cm,bindingoffset=0cm]{geometry}
\usepackage[utf8x]{inputenc}
\usepackage[english,russian]{babel}
\usepackage{cmap}
\usepackage{amssymb}
\usepackage{amsmath}
\usepackage{url}
\usepackage{pifont}
\usepackage{tikz}
\usepackage{verbatim}

\usetikzlibrary{shapes,arrows}
\usetikzlibrary{positioning,automata}
\tikzset{every state/.style={minimum size=0.2cm},
initial text={}
}


\newenvironment{myauto}[1][3]
{
  \begin{center}
    \begin{tikzpicture}[> = stealth,node distance=#1cm, on grid, very thick]
}
{
    \end{tikzpicture}
  \end{center}
}


\begin{document}
\begin{center} {\LARGE Формальные языки} \end{center}

\begin{center} \Large домашнее задание до 23:59 05.03 \end{center}
\bigskip

\begin{enumerate}
  \item Доказать или опровергнуть утверждение: произведение двух минимальных автоматов всегда дает минимальный автомат (рассмотреть случаи для пересечения, объединения и разности языков).
  \\
  (a) Объединение: \\
  Возьмём два автомата. Первый принимает все слова, в которых чётное количество букв $a$, а второй, в которых нечетное количество: \\
  \begin{myauto}
  	\node[state,initial, accepting]   (A)  {$A$};
  	
  	\node[state] (B) [right=of A] {$B$};
  	
  	\path[->]
  	(A) edge  [bend right=10] node [below] {$a$}  (B)
  	  
  	(B) edge  [bend right=10] node  [above] {$a$}  (A)
  	
  	;
  \end{myauto}

  \begin{myauto}
	\node[state, accepting]   (Y)              {$Y$};
	\node[state,initial]      (X) [left = of Y] {$X$};
	
	
	\path[->]
	
	(Y) edge  [bend right=10] node  [above] {$a$}  (X)
	(X) edge  [bend right=10] node  [below] {$a$}  (Y)
	
	;
\end{myauto}

Тогда их объединением будет : 

  \begin{myauto}
	\node[state, accepting]   (AY)              {$AY$};
	
	\node[state,initial, accepting]    (AX) [left = of AY] {$AX$};
	\node[state,accepting]    (BY) [below = of AY] {$BY$};
	\node[state]    (BX) [left = of BY] {$BX$};
	
	\path[->]
	(BY) edge  [bend right=10] node  [above] {$a$}  (AX)
	(AX) edge  [bend right=10] node  [below] {$a$}  (BY)
	;
\end{myauto}
  
  Но их объединение - это вообще все строки из букв $a$, поэтому минимальный автомат - это просто петля \\
  \\
  (b) Пересечение: \\
  У нас был пример на паре про делимость на 15. Если просто пересечь автоматы с делимостью на 3 и на 5, то получится 6 вершин, но на паре мы построили пример, где было только 4 вершины. \\
  \\
  Ну или можно взять предыдущий пример, в нём пересечение языков - это пустое множество, а в автомате пересечения 3 вершины (если удалить BX) \\
  (с) Разность: \\
  Можно например взять автоматы для $a^*b^*$ и $a^+b^*$ \\
  Тогда если рисовать полные автоматы, то в первом будет 2 вершины + дьявольская, во втором 3 вершины плюс дьявольская \\
    \begin{myauto}
  	\node[state, accepting]   (B)              {$B$};
  	
  	\node[state,initial, accepting]    (A) [left = of B] {$A$};
  	\node[state]    (D1) [right = of B] {$D_1$};
  	
  	\path[->]
  	(A) edge [loop above] node [above] {$a$}    ()
  	(A) edge   node  [above] {$b$}  (B)
  	(B) edge [loop above] node [above] {$b$}    ()
  	(B) edge   node  [above] {$a$}  (D1)
  	;
  \end{myauto}

    \begin{myauto}
	\node[state, accepting]   (Y)              {$Y$};
	
	\node[state,initial]    (X) [left = of Y] {$X$};
	\node[state, accepting]    (Z) [right = of Y] {$Z$};
	\node[state]    (D2) [below = of Y] {$D_2$};
	
	\path[->]
	(Y) edge [loop above] node [above] {$a$}    ()
	(X) edge   node  [above] {$a$}  (Y)
	(Z) edge [loop above] node [above] {$b$}    ()
	(Y) edge   node  [above] {$b$}  (Z)
	(Z) edge   node  [above] {$a$}  (D2)
	(X) edge   node  [above] {$b$}  (D2)
	;
\end{myauto}
  Разность первого языка со вторым - это просто $b^*$ - то есть просто одна петля, но сразу видно, что в произведение автоматов будет 12 вершин и даже после сокращения лишних это будет не петля, например потому что там как минимум 2 терминальные вершины - AX и BX\\
  \\
  Или можно сделать по-другому. Возьмём 2 равных языка, например $a^+b^+$, тогда их разность это пустое множество, но автомат произведения явно содержит хотя бы 2 вершины \\
  \\
  
  \item Для регулярного выражения:
   \[ (a \mid b)^+ (aa \mid bb \mid abab \mid baba)^* (a \mid b)^+\]
  Построить эквивалентные:
  \begin{enumerate}
    \item Недетерминированный конечный автомат
    \item Недетерминированный конечный автомат без $\varepsilon$-переходов
    \item Минимальный полный детерминированный конечный автомат
  \end{enumerate}
	Заметим, что из регулярного выражения можно выкинуть среднюю скобку. И будет тот же самый язык, так как: \\
	Если слово $S$ принадлежало \[ (a \mid b)^+ (aa \mid bb \mid abab \mid baba)^* (a \mid b)^+\] то его очевидно можно сконструировать с помощью \[ (a \mid b)^+ (a \mid b)^+\] \\
	Нужно просто среднюю часть $ (aa \mid bb \mid abab \mid baba)^* $ 'занести' в $ (a \mid b)^+$ (мы же там можем произвольное количество букв a и b использовать) \\
	И обратно, если слово $S$ принадлежало \[ (a \mid b)^+ (a \mid b)^+\] то его можно составить и с помощью\[ (a \mid b)^+ (aa \mid bb \mid abab \mid baba)^* (a \mid b)^+\] Так как средняя часть со звёздочкой, то можно просто повторить её 0 раз. \\
	\\
	(a) (b) Недетерминированный конечный автомат без $\varepsilon$-переходов для \[ (a \mid b)^+ (a \mid b)^+\]
	\begin{myauto}
		\node[state]           (q_1)                {$q_1$};
		\node[state,initial]   (q_0) [left=of  q_1] {$q_0$};
		
		\node[state,accepting] (q_2) [right=of q_1] {$q_2$};
		
		\path[->](q_0) edge     node [above] {$a, b$}       (q_1)
		(q_1) edge [loop above] node [above] {$a, b$}    ()
			  edge              node [above] {$a, b$}    (q_2)
		(q_2) edge [loop above] node [above] {$a, b$}    ()

		;
	\end{myauto}	
    (c) Минимальный полный детерминированный конечный автомат
    \begin{myauto}
    	\node[state]           (q_1)                {$q_1$};
    	\node[state,initial]   (q_0) [left=of  q_1] {$q_0$};
    	
    	\node[state,accepting] (q_2) [right=of q_1] {$q_2$};
    	
    	\path[->](q_0) edge     node [above] {$a, b$}       (q_1)
    	(q_1) edge              node [above] {$a, b$}    (q_2)
    	(q_2) edge [loop above] node [above] {$a, b$}    ()
    	
    	;
    \end{myauto}
    Понятно, что меньше 3 вершин быть не может, так как у нас есть две скобочки со знаком $+$, а значит во всех словах есть как минимум 2 буквы, а значит нужно 3 вершины. \\
  
  \item Построить регулярное выражение, распознающее тот же язык, что и автомат:
  \begin{myauto}
    \node[state]           (q_2)                {$q_2$};
    \node[state,initial]   (q_0) [left=of  q_2] {$q_0$};
    \node[state]           (q_1) [above=of q_2] {$q_1$};
    \node[state]           (q_3) [below=of q_2] {$q_3$};
    \node[state,accepting] (q_4) [right=of q_2] {$q_4$};

    \path[->] (q_0) edge [loop above] node [above] {$a, b, c$} ()
                    edge              node [above] {$a$}       (q_1)
                    edge              node [above] {$b$}       (q_2)
                    edge              node [above] {$c$}       (q_3)
              (q_1) edge [loop above] node [above] {$b, c$}    ()
                    edge              node [above] {$a$}       (q_4)
              (q_2) edge [loop above] node [above] {$a, c$}    ()
                    edge              node [above] {$b$}       (q_4)
              (q_3) edge [loop above] node [above] {$a, b$}    ()
                    edge              node [above] {$c$}       (q_4)
    ;
  \end{myauto}
   Вершина $q_0$ даёт нам скобочку $(a|b|c)^*$ \\
   А дальше у нас есть 3 пути до $q_4$, перебираем каждый из них: $a(b|c)^*a$, $b(a|c)^*b$, $c(a|b)^*c$ \\
   Итого: \[ (a|b|c)^* ((a(b|c)^*a) \ | \ (b(a|c)^*b) \  | \ (c(a|b)^*c)) \]
  
  \item Определить, является ли автоматным язык $\{ \omega \omega^r \mid \omega \in \{ 0, 1 \}^* \}$. Если является --- построить автомат, иначе --- доказать.
  \\
  Пусть он автоматный. Тогда выполняется условие леммы о накачке. Возьмём $w = 1^n 0 0 1 ^ n$, где $n$ - из леммы \\
  $w$ лежит в языке, так как $\omega = 1^n 0$ \\
  Так как $|w| > n$, то существует разбиение $xyz = w$, причём $y$ не пустой. Так как по условию леммы $|xy| < n$, то $y$ точно содержится в первых n единичках. Поэтому мы можем накачать $w$ повторив $y$ $k >= 2$ раза. Тогда будем получать слова вида $1^{n+c}001^n$ \\
  Такие слова точно не лежат в языке, так как у нас всего два нуля. А значит омега должна содержать ровно один ноль. А значит относительно этих нулей слово должно быть симметричным, но $c + n > n$, так как $y$ не пуст \\
  Противоречие с тем, что язык регулярный.
  \\
  
  \item Определить, является ли автоматным язык $\{ u a a v \mid u, v \in \{ a, b \}^* , |u|_b \geq |v|_a \}$. Если является --- построить автомат, иначе --- доказать.
  \\
  Пусть он автоматный. Тогда выполняется условие леммы о накачке. Возьмём $w = b^n a a (b a) ^ n$, где $n$ - из леммы \\
  $w$ лежит в языке, так как $u = b^n , \ v = (ba)^n$, $w = uaav$ и $|u|_b = |v|_a = n$  \\
Так как $|w| > n$, то существует разбиение $xyz = w$, причём $y$ не пустой. Так как по условию леммы $|xy| < n$, то $y$ точно содержит только буквы $b$ и причём точно содержит хотя бы одну. \\
То есть если мы возьмём $xy^0z$, то количество букв $b$ в $u$ строго уменьшиться. При этом разбиение на $u$ и $v$ однозначное, так как в нашем слове есть всего одно место, где встречаются две буквы $a$ подряд : $bbb...b \ aa \ bababa...ba$ \\
А то есть теперь $|u|_b < n = |v|_a$. \\
Полученное слово по условию леммы должно лежать в языке, но он не лежит. Противоречие с тем, что язык регулярный.
  \\
\end{enumerate}

\newpage

\begin{center}
  \Large{Пример применения алгоритма минимизации}
\end{center}

\bigskip

Минимизируем данный автомат:

\begin{center}
  \begin{tikzpicture}[> = stealth,node distance=3cm, on grid]
    \node[state]           (q_2)                      {C};
    \node[state,initial]   (q_0) [above left=of q_2]  {A};
    \node[state]           (q_1) [below left=of q_2]  {B};
    \node[state]           (q_3) [right=of q_2]       {D};
    \node[state]           (q_4) [above right=of q_3] {E};
    \node[state,accepting] (q_5) [below right=of q_3] {F};
    \node[state,accepting] (q_6) [above right=of q_5] {G};

    \path[->] (q_0) edge [bend left=15]  node [right] {$1$} (q_1)
                    edge                 node [above] {$0$} (q_2)
              (q_1) edge [bend left=15]  node [left]  {$1$} (q_0)
                    edge                 node [below] {$0$} (q_2)
              (q_2) edge [bend right=15] node [below] {$1$} (q_3)
                    edge [bend left=15]  node [above] {$0$} (q_3)
              (q_3) edge                 node [below] {$1$} (q_5)
                    edge                 node [above] {$0$} (q_4)
              (q_4) edge                 node [above] {$1$} (q_6)
                    edge                 node [right] {$0$} (q_5)
              (q_5) edge [loop below]    node         {$1$} ()
                    edge [loop left]     node         {$0$} ()
              (q_6) edge                 node [below] {$1$} (q_5)
                    edge [loop right]    node         {$0$} ();
  \end{tikzpicture}
\end{center}

Автомат полный, в нем нет недостижимых вершин --- продолжаем.

Строим обратное $\delta$ отображение.

\begin{tabular}{c|c|c}
$\delta^{-1}$ & 0 & 1 \\ \hline
A & --- & B \\
B & --- & A \\
C & A B & --- \\
D & C & C \\
E & D & --- \\
F & E F & D F G \\
G & G & E
\end{tabular}

Отмечаем в таблице и добавляем в очередь пары состояний, различаемых словом $\varepsilon$: все пары, один элемент которых --- терминальное состояние, а второй --- не терминальное состояние. Для данного автомата это пары

$(A, F), (B, F), (C, F), (D, F), (E,F), (A, G), (B, G), (C, G), (D, G), (E, G)$

Дальше итерируем процесс определения неэквивалентных состояний, пока очередь не оказывается пуста.

$(A, F)$ не дает нам новых неэквивалентных пар. Для $(B, F)$ находится 2 пары: $(A, D), (A, G)$. Первая пара не отмечена в таблице --- отмечаем и добавляем в очередь. Вторая пара уже отмечена в таблице, значит, ничего делать не надо. Переходим к следующей паре из очереди. Итерируем дальше, пока очередь не опустошится.

Результирующая таблица (заполнен только треугольник, потому что остальное симметрично) и порядок добавления пар в очередь.

\begin{tabular}{c|cc|cc|cc|c}
& A & B & C & D & E & F & G \\ \hline
A &&&&&&& \\
B &&&&&&& \\ \hline
C & \checkmark & \checkmark &&&&& \\
D & \checkmark & \checkmark & \checkmark &&&& \\ \hline
E & \checkmark & \checkmark & \checkmark & \checkmark &&& \\
F & \checkmark & \checkmark & \checkmark & \checkmark & \checkmark && \\ \hline
G & \checkmark & \checkmark & \checkmark & \checkmark & \checkmark && \\
\end{tabular}

Очередь:

$
(A, F), (B, F), (C, F), (D, F), (E,F), (A, G), (B, G), (C, G), (D, G), (E, G),
$

$
(B, D), (A, D), (A, E), (B, E), (C, E), (C, D), (D, E), (A,C), (B, C))
$

В таблице выделились классы эквивалентных вершин: $\{A, B\}, \{C\}, \{D\}, \{E\}, \{F,G\}$. Остается только нарисовать результирующий автомат с вершинами-классами. Переходы добавляются тогда, когда из какого-нибудь состояния первого класса есть переход в какое-нибудь состояние второго класса. Минимизированный автомат:

\begin{center}
  \begin{tikzpicture}[> = stealth,node distance=3cm, on grid]
    \node[state,initial]   (q_01)                     {AB};
    \node[state]           (q_2)  [right=of q_01]      {C};
    \node[state]           (q_3)  [right=of q_2]       {D};
    \node[state]           (q_4)  [above right=of q_3] {E};
    \node[state,accepting] (q_56) [below right=of q_3] {FG};

    \path[->] (q_01) edge [loop above]    node [above] {$1$} ()
                     edge                 node [above] {$0$} (q_2)
              (q_2)  edge [bend right=15] node [below] {$1$} (q_3)
                     edge [bend left=15]  node [above] {$0$} (q_3)
              (q_3)  edge                 node [below] {$1$} (q_56)
                     edge                 node [above] {$0$} (q_4)
              (q_4)  edge [bend right=15] node [left]  {$1$} (q_56)
                     edge [bend left=15]  node [right] {$0$} (q_56)
              (q_56) edge [loop below]    node         {$1$} ()
                     edge [loop left]     node         {$0$} ();
  \end{tikzpicture}
\end{center}

\end{document}
