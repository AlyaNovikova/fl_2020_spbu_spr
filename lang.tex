\documentclass[12pt]{article}

\usepackage{cmap}
\usepackage[T2A]{fontenc}
\usepackage[utf8]{inputenc}
\usepackage[russian]{babel}
\usepackage{graphicx}
\usepackage{amsthm,amsmath,amssymb}
\usepackage[russian,colorlinks=true,urlcolor=red,linkcolor=blue]{hyperref}
\usepackage{enumerate}
\usepackage{datetime}
\usepackage{fancyhdr}
\usepackage{lastpage}
\usepackage{color}
\usepackage{verbatim}
\usepackage{tikz}
\usepackage{epstopdf}
\usepackage{xifthen}

\def\NAME{}
\def\DATE{}
\def\CURNO{\NO\t{7}}
 
\parskip=0em
\parindent=0em

\sloppy
\voffset=-20mm
\textheight=235mm
\hoffset=-25mm
\textwidth=180mm
\headsep=12pt
\footskip=20pt

\setcounter{page}{0}
\pagestyle{empty}

\DeclareSymbolFont{extraup}{U}{zavm}{m}{n}
\DeclareMathSymbol{\heart}{\mathalpha}{extraup}{86}
\newcommand{\N}{\mathbb{N}}   \newcommand{\R}{\mathbb{R}}   \newcommand{\Z}{\mathbb{Z}}   \def\EPS{\varepsilon}         \def\SO{\Rightarrow}          \def\EQ{\Leftrightarrow}      \def\t{\texttt}               \def\c#1{{\rm\sc{#1}}}        \def\O{\mathcal{O}}           \def\NO{\t{\#}}               \renewcommand{\le}{\leqslant} \renewcommand{\ge}{\geqslant} \def\XOR{\text{ {\raisebox{-2pt}{\ensuremath{\Hat{}}}} }}
\newcommand{\q}[1]{\langle #1 \rangle}               \newcommand\URL[1]{{\footnotesize{\url{#1}}}}        \newcommand{\sfrac}[2]{{\scriptstyle\frac{#1}{#2}}}  \newcommand{\mfrac}[2]{{\textstyle\frac{#1}{#2}}}    \newcommand{\score}[1]{{\bf\color{red}{(#1)}}}

\def\makeparindent{\hspace*{\parindent}}
\def\up{\vspace*{-0.3em}}
\def\down{\vspace*{0.3em}}
\def\LINE{\vspace*{-1em}\noindent \underline{\hbox to 1\textwidth{{ } \hfil{ } \hfil{ } }}}


\lhead{Александра Новикова}
\chead{}
\rhead{\DATE}
\renewcommand{\headrulewidth}{0.4pt}

\lfoot{}
\cfoot{\thepage\t{/}\pageref*{LastPage}}
\rfoot{}
\renewcommand{\footrulewidth}{0.4pt}

\newenvironment{MyList}[1][4pt]{
  \begin{enumerate}[1.]
  \setlength{\parskip}{0pt}
  \setlength{\itemsep}{#1}
}{       
  \end{enumerate}
}
\newenvironment{InnerMyList}[1][0pt]{
  \vspace*{-0.5em}
  \begin{enumerate}[a)]
  \setlength{\parskip}{#1}
  \setlength{\itemsep}{0pt}
}{
  \end{enumerate}
}

\newcommand{\Section}[1]{
  \refstepcounter{section}
  \addcontentsline{toc}{section}{\arabic{section}. #1} 
{\LARGE \bf #1} 
  \vspace*{1em}
  \makeparindent\unskip
}
\newcommand{\Subsection}[1]{
  \refstepcounter{subsection}
  \addcontentsline{toc}{subsection}{\arabic{section}.\arabic{subsection}. #1} 
  {\Large \bf \arabic{section}.\arabic{subsection}. #1} 
  \vspace*{0.5em}
  \makeparindent\unskip
}

\newlength{\ShiftLength}\setlength{\ShiftLength}{1.8em}
\newcommand{\leftLabel}[2][]{\ifthenelse{\equal{#1}{}}{{\hspace*{-\ShiftLength}\makebox[0pt][r]{\color{black}{#2}}\hspace*{\ShiftLength}}}{{\hspace*{-\ShiftLength}\makebox[0pt][r]{\color{#1}{#2}}\hspace*{\ShiftLength}}}}

\newenvironment{code}{
  \VerbatimEnvironment

  \vspace*{-0.5em}
  \begin{minted}{c}}{
  \end{minted}
  \vspace*{-0.5em}

}

\newenvironment{smallformula}{
 
  \vspace*{-0.8em}
}{
  \vspace*{-1.2em}
  
}
\newenvironment{formula}{
 
  \vspace*{-0.4em}
}{
  \vspace*{-0.6em}
  
}

\definecolor{dkgreen}{rgb}{0,0.6,0}
\definecolor{brown}{rgb}{0.5,0.5,0}
\newcommand{\red}[1]{{\color{red}{#1}}}
\newcommand{\dkgreen}[1]{{\color{dkgreen}{#1}}}

\begin{document}

\pagestyle{fancy}


 Давайте сконструируем язык \textbf{L} через \textbf{ПРЯМУЮ (префиксную) польскую запись}.\\
 \\
 Зададим алфавит $\Sigma_0 = \{a,b,...,z,A,B,...,Z,0,...,9, \_ \}$ - буквы, цифры и нижнее подчёркивание \\
 Зададим наш терминальный алфавит $\Sigma = \Sigma_0 \cup \{$ \ , ... , \ $ \} $, то есть это $\Sigma_0$ и все пробельные символы \\
 Нетерминальный алфавит $N =\{s_0, E_0, V_0, D_0,$\textvisiblespace$  \}$, где $s_0$ - стартовое состояние, $E_0$ - это Expr, $V_0$ - это Var и Ident (в данном случае это одно и тоже), $D_0$ - это числа, \textvisiblespace - все пробельные символы (причём давайте считать, что это всевозможные их комбинации, то есть может быть несколько пробелов подряд) \\
 !Важное замечание про пробелы! : все токены разделены пробелами,о есть ключевые слова, переменные, операторы и числа. Далее в грамматике указаны пробелы только внутри одной конструкции, но между конструкциями пробелы тоже обязаны быть. \\ 
 \\
 В нашем языке есть ключевые слова \textbf{If, While, Assign, Read, Write и Seq, Nop} \\
 (Nop - пустая инструкция) \\
 Для каждого из них введём правило: \\ 
 \begin{itemize}
 	\item $s_0 \to If $\textvisiblespace$ E_0 $\textvisiblespace$ s_0 $\textvisiblespace$ s_0$ \\
 	(Порядок веток прямой, соответственно если условие $E_0$ выполнится, то будет исполнятся первый $s_0$, иначе второй) 
 	\item $s_0 \to While $\textvisiblespace$ E_0 $\textvisiblespace$ s_0 $
 	\item $s_0 \to Assign $\textvisiblespace$V_0 $\textvisiblespace$ E_0 $
 	\item $s_0 \to V_0 $\textvisiblespace$ Read$ 
 	\item $s_0 \to Write $\textvisiblespace$  E_0 $
 	\item $s_0 \to Seq $\textvisiblespace$  s_0 $\textvisiblespace$ s_0 $ \\
 	(Порядок прямой, соответственно вначале идёт первое действие, затем второе) 
 	\item $s_0 \to Nop$ \\ (пустая инструкция)
 \end{itemize}

		
\vspace{\baselineskip}
 
 
 \textbf{Выражения}: \\
 Правила для бинарных операторов (порядок везде прямой): \\
 \begin{itemize}
	\item $E_0 \to - $\textvisiblespace$ E_0 $\textvisiblespace$ E_0 $ \\
	(Из первого $E_0$ вычитаем второе $E_0$) 
	\item $E_0 \to + $\textvisiblespace$ E_0 $\textvisiblespace$ E_0 $ \\
	...
	\item $E_0 \to \&\& $\textvisiblespace$ E_0 $\textvisiblespace$ E_0 $
	\item $E_0 \to || $\textvisiblespace$ E_0 $\textvisiblespace$ E_0 $

\end{itemize}
  
  Правила для унарных операторов (унарный минус и отрицание): \\
  
  \begin{itemize}
 	\item $E_0 \to  - -  $\textvisiblespace$ E_0$ (унарный минус)
 	\item $E_0 \to  !  $\textvisiblespace$ E_0$ (отрицание)
 \end{itemize}
   
   Правила для чисел и переменных: \\
  
  \begin{itemize}
	\item $E_0 \to D_0$
	\item $E_0 \to V_0$
\end{itemize}

\vspace{\baselineskip}
\vspace{\baselineskip}

 \textbf{Переменные и идентификаторы} - это строки, которые начинаются обязательно с буквы либо нижнего подчёркивания и состоят из символов $\Sigma_0$, причём обязательно \textbf{НЕ совпадают} с ключевыми словами  \\
\begin{itemize}
	\item $V_0 \to (a | b | ... |\_ |\_a| ... |aa|ab|...|Alya\_0| .......|Fkb1aa1c\_d3|...)\ \Sigma_0^*$
\end{itemize}
(Перечисляем все корректные слова длины < 15 \textbf{(кроме ключевых слов)} и дальше добавляем что угодно, то есть $\Sigma_0^*$) \\

\textbf{Числа}: 
\begin{itemize}
	\item $D_0 \to (0|1|...|9)^+$
\end{itemize}

\textbf{Пробелы}: \\

    \begin{itemize}
  	\item \textvisiblespace $\to (\backslash t \ | \ \backslash n \ | ...)^+$
  	(все пробельные символы и любые их комбинации)
  \end{itemize}


\end{document}